\documentclass{report}
\usepackage{graphicx}
\usepackage[a4paper, margin=0.5in]{geometry}
\usepackage{subcaption}
%\usepackage{printlen}
\usepackage{pdfpages}
%\uselengthunit{cm}
%\usepackage{import}
%\usepackage{xifthen}
\usepackage{ifthen}
\usepackage{pgf}
\usepackage{tikz}
%\usepackage{comment}
\usepackage{intcalc}
\usepackage{xcolor} % Required for specifying custom colours
\definecolor{grey}{rgb}{0.9,0.9,0.9} % Colour of the box surrounding the title

\newlength\imageheight
\newlength\imagewidth

\newcommand{\includegraphicsmaybe}[1]{\IfFileExists{#1}{\includegraphics[width=\linewidth]{#1}}{\includegraphics[width=\linewidth]{not_found.pdf}}}
\newcommand\dateofscans{04-04-23} %Enter the date on which the scans were made (e.g. 06-13-22), assuming eyescans were properly filed

\usepackage{booktabs}

\usepackage{hyperref}

\setlength{\parindent}{0pt}

\newcounter{includepdfpage}
\newcounter{currentpagecounter}
\newcommand{\addlabelstoallincludedpages}[1]{
   \refstepcounter{includepdfpage}
   \stepcounter{currentpagecounter}
   \label{#1.\thecurrentpagecounter}}
\newcommand{\modifiedincludepdf}[3]{
    \setcounter{currentpagecounter}{0}
    \includepdf[pages=#1,pagecommand=\addlabelstoallincludedpages{#2}]{#3}}

\begin{document}

%----------------------------------------------------------------------------------------
%TITLE PAGE
%----------------------------------------------------------------------------------------

\begin{titlepage} % Suppresses displaying the page number on the title page and the subsequent page counts as page 1
  
  %------------------------------------------------
  %Grey title box
  %------------------------------------------------
  
  \colorbox{grey}{
    \parbox[t]{0.93\textwidth}{ % Outer full width box
      \parbox[t]{0.91\textwidth}{ % Inner box for inner right text margin
        \raggedleft % Right align the text
        \fontsize{80pt}{40pt}\selectfont % Title font size, the first argument is the font size and the second is the line spacing, adjust depending on title length
        \vspace{0.7cm} % Space between the start of the title and the top of the grey box

        CM209 Apollo 214\\
        Serial Link Test Report\\
        Date of Scans: \dateofscans\\
        
        \vspace{0.7cm} % Space between the end of the title and the bottom of the grey box
        }
      }
    }
  
  \vfill % Space between the title box and author information
  
  %------------------------------------------------
  %Author name and information
  %------------------------------------------------
  
  \parbox[t]{0.93\textwidth}{ % Box to inset this section slightly
    \raggedleft % Right align the text
    \large % Increase the font size\large % Increase the font size
    {\Large Rui Zou}\\[4pt] % Extra space after name
    CERN\\
    Cornell University\\[4pt] % Extra space before URL
    
    \hfill\rule{0.2\linewidth}{1pt}% Horizontal line, first argument width, second thickness
    }
  
\end{titlepage}

\phantomsection 
\hypertarget{MyToc}{}  % Make an anchor to the toc
\tableofcontents
\newpage

%\chapter{Initial Considerations}

%\section{Eye diagram mask}

%\begin{figure}[h!]
%\centering
%\includegraphics[width=0.70\linewidth]{ieeemask.png}
%\caption{Transmitter eye mask definition}
%\label{fig:ieeemask}
%\end{figure}

%Figure \ref{fig:ieeemask} shows the eye mask parameters {X1, X2, X3, Y1, Y2, Y3} = {0.25, 0.40, 0.45, 0.25, 0.28, 0.80} extracted from the Table 68-3---10GBASE-LRM transmit characteristics part of the IEEE Std 802.3-2015 SECTION 5. 


%REAL START OF PDF
%%%%%%%%%%%%%%%%%%%%%%%%%%%%%%%%%%%%%%%%%%%%%%%%%%%%%
\chapter{Links Between xcvu13p\_0 to xcvu13p\_1 Overview (1 of 2)}
\hyperlink{MyToc}{Return to Table of Contents} \newpage

%links between xcvu13p_0 & xcvu13p_1
\begin{figure}[h] % ``[t!]'' placement specifier just for this example
\foreach \number [count=\nxtrow from int(1),count=\i from int(\thepage+2),evaluate=\number as \Rx using int(63-\number)] in {4,...,15,20,21,...,35}{ %numbers here can be modified based on links scanned
\ifthenelse{\nxtrow=4 \or \nxtrow=8 \or \nxtrow=12 \or \nxtrow=16 \or \nxtrow=20 \or \nxtrow=24 \or \nxtrow=28}
{\begin{subfigure}{0.25\textwidth}
\hyperlink{page.\i}{\includegraphicsmaybe{../../scans/CM209/\dateofscans/DFE/eyescan_DFE_adjust_X1Y\number(xcvu13p_0)_to_X1Y\Rx(xcvu13p_1).png}}
\end{subfigure}

}{\begin{subfigure}{0.25\textwidth}
\hyperlink{page.\i}{\includegraphicsmaybe{../../scans/CM209/\dateofscans/DFE/eyescan_DFE_adjust_X1Y\number(xcvu13p_0)_to_X1Y\Rx(xcvu13p_1).png}}
\end{subfigure}\hspace*{\fill}}}
\caption{Links of xcvu13p\_0 to xcvu13p\_1 Overview (1 of 2)} \label{links_xcvu13p_0_to_xcvu13p_1_overview1}
\end{figure}

\chapter{Links of xcvu13p\_0 to xcvu13p\_1 Details (1 of 2)} \hyperlink{MyToc}{Return to Table of Contents} \newpage
\foreach \n [evaluate=\n as \Rx using int(63-\n)] in {4,...,15,20,21,...,35}{
\modifiedincludepdf{-}{X1Y\n(xcvu13p_0)_to_X1Y\Rx(xcvu13p_1)}{../../scans/CM209/\dateofscans/DFE/eyescan_DFE_adjust_X1Y\n(xcvu13p_0)_to_X1Y\Rx(xcvu13p_1).pdf}
}

%%%%%%%%%%%%%%%%%%%%%%%%%%%%%%%%%%%%%%%%%%%%%%%%%%%%%
\chapter{Links Between xcvu13p\_0 to xcvu13p\_1 Overview (2 of 2)}
\hyperlink{MyToc}{Return to Table of Contents} \newpage

%links between xcvu13p_0 & xcvu13p_1
\begin{figure}[h] % ``[t!]'' placement specifier just for this example
\foreach \number [count=\nxtrow from int(1),count=\i from int(\thepage+2),evaluate=\number as \Rx using int(63-\number)] in {36,...,59}{ %numbers here can be modified based on loopback links scanned
\ifthenelse{\nxtrow=4 \or \nxtrow=8 \or \nxtrow=12 \or \nxtrow=16 \or \nxtrow=20 \or \nxtrow=24 \or \nxtrow=28}
{\begin{subfigure}{0.25\textwidth}
\hyperlink{page.\i}{\includegraphicsmaybe{../../scans/CM209/\dateofscans/DFE/eyescan_DFE_adjust_X1Y\number(xcvu13p_0)_to_X1Y\Rx(xcvu13p_1).png}}
\end{subfigure}

}{\begin{subfigure}{0.25\textwidth}
\hyperlink{page.\i}{\includegraphicsmaybe{../../scans/CM209/\dateofscans/DFE/eyescan_DFE_adjust_X1Y\number(xcvu13p_0)_to_X1Y\Rx(xcvu13p_1).png}}
\end{subfigure}\hspace*{\fill}}}
\caption{Links of xcvu13p\_0 to xcvu13p\_1 Overview (2 of 2)} \label{links_xcvu13p_0_to_xcvu13p_1_overview2}
\end{figure}

\chapter{Links of xcvu13p\_0 to xcvu13p\_1 Details (2 of 2)} \hyperlink{MyToc}{Return to Table of Contents} \newpage
\foreach \n [evaluate=\n as \Rx using int(63-\n)] in {36,...,59}{
\modifiedincludepdf{-}{X1Y\n(xcvu13p_0)_to_X1Y\Rx(xcvu13p_1)}{../../scans/CM209/\dateofscans/DFE/eyescan_DFE_adjust_X1Y\n(xcvu13p_0)_to_X1Y\Rx(xcvu13p_1).pdf}
}

%%%%%%%%%%%%%%%%%%%%%%%%%%%%%%%%%%%%%%%%%%%%%%%%%%%%%
\chapter{Links From xcvu13p\_1 to xcvu13p\_0 Overview (1 of 2)}
\hyperlink{MyToc}{Return to Table of Contents} \newpage

%links between xcvu13p_0 & xcvu13p_1
\begin{figure}[h] % ``[t!]'' placement specifier just for this example
\foreach \number [count=\nxtrow from int(1),count=\i from int(\thepage+2),evaluate=\number as \Rx using int(63-\number)] in {59,...,48,43,42,...,28}{ %{4,...,31}{ %numbers here can be modified based on links scanned
\ifthenelse{\nxtrow=4 \or \nxtrow=8 \or \nxtrow=12 \or \nxtrow=16 \or \nxtrow=20 \or \nxtrow=24 \or \nxtrow=28}
{\begin{subfigure}{0.25\textwidth}
\hyperlink{page.\i}{\includegraphicsmaybe{../../scans/CM209/\dateofscans/DFE/eyescan_DFE_adjust_X1Y\number(xcvu13p_1)_to_X1Y\Rx(xcvu13p_0).png}}
\end{subfigure}

}{\begin{subfigure}{0.25\textwidth}
\hyperlink{page.\i}{\includegraphicsmaybe{../../scans/CM209/\dateofscans/DFE/eyescan_DFE_adjust_X1Y\number(xcvu13p_1)_to_X1Y\Rx(xcvu13p_0).png}}
\end{subfigure}\hspace*{\fill}}}
\caption{Links of xcvu13p\_1 to xcvu13p\_0 Overview (1 of 2)} \label{links_xcvu13p_1_to_xcvu13p_0_overview1}
\end{figure}

\chapter{Links of xcvu13p\_1 to xcvu13p\_0 Details (1 of 2)} \hyperlink{MyToc}{Return to Table of Contents} \newpage
\foreach \n [evaluate=\n as \Rx using int(63-\n)] in {59,...,48,43,42,...,28}{
\modifiedincludepdf{-}{X1Y\n(xcvu13p_1)_to_X1Y\Rx(xcvu13p_0)}{../../scans/CM209/\dateofscans/DFE/eyescan_DFE_adjust_X1Y\n(xcvu13p_1)_to_X1Y\Rx(xcvu13p_0).pdf}
}

%%%%%%%%%%%%%%%%%%%%%%%%%%%%%%%%%%%%%%%%%%%%%%%%%%%%%
\chapter{Links From xcvu13p\_1 to xcvu13p\_0 Overview (2 of 2)}
\hyperlink{MyToc}{Return to Table of Contents} \newpage

%links between xcvu13p_0 & xcvu13p_1
\begin{figure}[h] % ``[t!]'' placement specifier just for this example
\foreach \number [count=\nxtrow from int(1),count=\i from int(\thepage+2),evaluate=\number as \Rx using int(63-\number)] in {27,...,4}{ %{32,...,43,48,49,...,59}{ %numbers here can be modified based on loopback links scanned
\ifthenelse{\nxtrow=4 \or \nxtrow=8 \or \nxtrow=12 \or \nxtrow=16 \or \nxtrow=20 \or \nxtrow=24 \or \nxtrow=28}
{\begin{subfigure}{0.25\textwidth}
\hyperlink{page.\i}{\includegraphicsmaybe{../../scans/CM209/\dateofscans/DFE/eyescan_DFE_adjust_X1Y\number(xcvu13p_1)_to_X1Y\Rx(xcvu13p_0).png}}
\end{subfigure}

}{\begin{subfigure}{0.25\textwidth}
\hyperlink{page.\i}{\includegraphicsmaybe{../../scans/CM209/\dateofscans/DFE/eyescan_DFE_adjust_X1Y\number(xcvu13p_1)_to_X1Y\Rx(xcvu13p_0).png}}
\end{subfigure}\hspace*{\fill}}}
\caption{Links of xcvu13p\_1 to xcvu13p\_0 Overview (2 of 2)} \label{links_xcvu13p_1_to_xcvu13p_0_overview2}
\end{figure}

\chapter{Links of xcvu13p\_1 to xcvu13p\_0 Details (2 of 2)} \hyperlink{MyToc}{Return to Table of Contents} \newpage
\foreach \n [evaluate=\n as \Rx using int(63-\n)] in {27,...,4}{
\modifiedincludepdf{-}{X1Y\n(xcvu13p_1)_to_X1Y\Rx(xcvu13p_0)}{../../scans/CM209/\dateofscans/DFE/eyescan_DFE_adjust_X1Y\n(xcvu13p_1)_to_X1Y\Rx(xcvu13p_0).pdf}
}


\end{document}
